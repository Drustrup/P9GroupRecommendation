% Make all warnings errors
%\renewcommand{\GenericWarning}[2]{\GenericError{#1}{#2}{}{This warning has been turned into a fatal error.}}
% Project related.
  % Project information.
  \newcommand{\projecttitle}{Group Recommendation Using \newline Voting as Mediator}
  \newcommand{\projecttheme}{Machine Intelligence, \newline Recommendation}
  \newcommand{\projectperiod}{Autumn semester 2016}
  \newcommand{\projectcopies}{3}
  \newcommand{\projectcompletion}{December 21, 2016}

  % Names.
  \newcommand{\groupname}{mi908e16}
  \newcommand{\supervisor}{Peter Dolog} % no \@ here!
  \newcommand{\groupmembersbyfirstname}[0]{%
  	Claus Nygaard Madsen\\
    Lasse Drustrup Christensen\\
    Lukas Nic Dalgaard
  }

% Custom styling commands.
  % Monospaced text.
  \newcommand*\justify{%
    \fontdimen2\font=0.4em% interword space
    \fontdimen3\font=0.2em% interword stretch
    \fontdimen4\font=0.1em% interword shrink
    \fontdimen7\font=0.1em% extra space
    \hyphenchar\font=`\-% allowing hyphenation
  }
  \newcommand{\mono}[1]{\texttt{\justify {#1}}}

  % Code mono
  \newcommand{\code}[1]{\sethlcolor{clcodeshade}\hl{\texttt{\justify{#1}}}} % Yes, this line is identical to the one in \mono, however the soul-package is fragile, and cannot see into another command.
  \soulregister{\justify}{1}

  % Chapter introductions.
  \newcommand{\chapterintro}[1]{#1}

  % Group X.
  \newcommand{\group}[1]{Group #1}

  % User story.
  \newcommand{\us}[1]{\emph{#1}}

  % Project names
  \newcommand{\gproject}[1]{#1}

  % Subproject names
  \newcommand{\gui}{GUI\@\xspace} \newcommand{\guititle}{GUI\xspace}
  \newcommand{\db}{DB\@\xspace} \newcommand{\dbtitle}{DB\xspace}
  \newcommand{\bd}{B\&D\@\xspace} \newcommand{\bdtitle}{B\&D\xspace}

  % CD paths with icon.
  \newcommand{\cdpath}[1]{%
    % Param1: path
    \hyperref[app:cd]{\raisebox{-0.28ex}{\includegraphics[height=0.85em]{cd}}\mono{/#1}}%
  }

  % Margin text.
  \newcommand{\margintext}[1]{\marginline{\textsf{\footnotesize #1}}}

  %\newcommand{\code}[1]{#1}

  % Todo.
  \newcommand{\todo}[1]{\fxnote{#1}}
  \FXRegisterAuthor{note}{notes}{\color{blue}note}
  \newcommand{\note}[1]{\notenote{#1}}

  % Dummy line.
  \newcommand{\dummy}{\todo{Lorem ipsum dolor sit amet, consectetur adipiscing elit.}}

% Document structure.
  % Sprints.
  \newcounter{sprintcounter}%\addtocounter{sprintcounter}{1}
  \newcommand{\sprint}[1]{%
    \cleardoublepage{}
    \stepcounter{sprintcounter}
    \part*{Sprint \arabic{sprintcounter}\addcontentsline{toc}{part}{Sprint \arabic{sprintcounter}}}%
  }

  % Chapter grouping.
  \newcommand{\chaptergroup}[1]{
    \clearpage{}
    \part*{#1}
    \addcontentsline{toc}{part}{#1}
  }

  % Document organization.
  \newenvironment{documentorganization}
    {\vspace{.5cm}\noindent\textbf{Report Organization}\quad The remainder of this report is organized in the following fashion:\begin{itemize}}
    {\end{itemize}}

  % Chapter organization.
  \newenvironment{chapterorganization}
    {\vspace{.5cm}\noindent\textbf{Chapter Organization}\quad This chapter is organized in the following fashion:\begin{itemize}}
    {\end{itemize}}

   % Abbreviation.
  \newenvironment{abbreviations}
    {\vspace{.5cm}\noindent\textbf{Chapter Abbreviations}\quad This chapter introduces the following abbreviations:\begin{addmargin}[\leftmargin]{0em}\begin{multicols}{2}\begin{description}[noitemsep, style=sameline]}
    {\end{description}\end{multicols}\end{addmargin}}

    % Dates.
    \newenvironment{dates}
    {\vspace{.5cm}\noindent\textbf{Project Dates}\quad \dummy:\begin{addmargin}[\leftmargin]{0em}\begin{multicols}{2}\begin{description}[noitemsep, style=sameline]}
    {\end{description}\end{multicols}\end{addmargin}}

% Figures.
  % Normal figure.
  \newcommand{\fig}[3]{
    \begin{figure}[tbp]
      \centering
      %\rule{\textwidth}{0.005in}
      \includegraphics[width=0.75\textwidth]{#1}
      \caption[#2]{#3}\label{fig:#1}
      %\rule{\textwidth}{0.005in}
    \end{figure}
  }

  % Scaled figure.
  \newcommand{\figscaled}[4]{
    \begin{figure}[tbp]
      \centering
      \includegraphics[scale=#4]{#1}
      \caption[#2]{#3}\label{fig:#1}
    \end{figure}
  }

  % Figure with custom width.
  \newcommand{\figcustomwidth}[4]{
    \begin{figure}[tbp]
      \centering
      \includegraphics[width=#4]{#1}
      \caption[#2]{#3}\label{fig:#1}
    \end{figure}
  }

% References.
  \newcommand{\appendixref}[1]{\hyperref[#1]{Appendix \ref*{#1}}}
  \newcommand{\chapterref}[1]{\hyperref[#1]{Chapter \ref*{#1}}}
  \newcommand{\sectionref}[1]{\hyperref[#1]{Section \ref*{#1}}}
  \newcommand{\figureref}[1]{\hyperref[#1]{Figure \ref*{#1}}}
  \newcommand{\tableref}[1]{\hyperref[#1]{Table \ref*{#1}}}
  \newcommand{\listingref}[1]{\hyperref[#1]{Listing \ref*{#1}}}
  \newcommand{\onpage}[1]{on \hyperref[#1]{page \pageref*{#1}}}
  \newcommand{\equationref}[1]{\hyperref[#1]{Equation \ref*{#1}}}
  \newcommand{\algorithmref}[1]{\hyperref[#1]{Algorithm \ref*{#1}}}
  %\newcommand{\sprintref}[1]{\hyperref[#1]{Sprint \ref*{#1}}}

% Column types for tabulars.
%\newcolumntype{L}[1]{>{\raggedright\let\newline\\\arraybackslash\hspace{0pt}}m{#1}}
%\newcolumntype{R}[1]{>{\raggedleft\let\newline\\\arraybackslash\hspace{0pt}}m{#1}}

% Fix todo notes with externalize
\let\oldTodo\todo
\renewcommand{\todo}[1]{\tikzexternaldisable{}\oldTodo{#1}\tikzexternalenable{}}

% Fixme stuff.
%\newcommand{\fillin}[1]{\fxnote*[inline]{#1}{Fill in: }}

% Vector command.
%\newcommand{\omatrix}[1]{\ensuremath{\boldsymbol{#1}}}

% A better plus minus sign.
\makeatletter
\newcommand{\gpm}{\mathbin{\mathpalette\@gpm\relax}}
\newcommand{\@gpm}[2]{\ooalign{%
  \raisebox{.1\height}{$#1+$}\cr
  \smash{\raisebox{-.6\height}{$#1-$}}\cr}}
\makeatother

% Math bold package
\usepackage{bm}

% A post it note % http://tex.stackexchange.com/questions/159679/sticky-notes-image

\usepackage{xparse}
\usepackage{fancypar}

\definecolor{myyellow}{RGB}{255,243,141}

\makeatletter
\pgfdeclareshape{stickyfront}{
  \inheritsavedanchors[from=rectangle] % this is nearly a rectangle
  \inheritanchorborder[from=rectangle]
  \inheritanchor[from=rectangle]{center}
  \inheritanchor[from=rectangle]{north}
  \inheritanchor[from=rectangle]{south}
  \inheritanchor[from=rectangle]{west}
  \inheritanchor[from=rectangle]{east}
  % ... and possibly more
  \backgroundpath{% this is new
  % store lower right in xa/ya and upper right in xb/yb
  \southwest \pgf@xa=\pgf@x \pgf@ya=\pgf@y
  \northeast \pgf@xb=\pgf@x \pgf@yb=\pgf@y
  % compute corner of ‘‘flipped page’’
  \pgf@xc=\pgf@xb \advance\pgf@xc by-20pt % this should be a parameter
  \pgf@yc=\pgf@ya \advance\pgf@yc by+20pt
  % construct main path
  \pgfpathmoveto{\pgfpoint{\pgf@xa}{\pgf@ya}}
  \pgfpathlineto{\pgfpoint{\pgf@xa}{\pgf@yb}}
  \pgfpathlineto{\pgfpoint{\pgf@xb}{\pgf@yb}}
  \pgfpathlineto{\pgfpoint{\pgf@xb}{\pgf@yc}}
  \pgfpathlineto{\pgfpoint{\pgf@xc}{\pgf@ya}}
  \pgfpathclose
  % add little corner
  \pgfpathmoveto{\pgfpoint{\pgf@xc}{\pgf@ya}}
  \pgfpathlineto{\pgfpoint{\pgf@xc}{\pgf@yc}}
  \pgfpathlineto{\pgfpoint{\pgf@xb}{\pgf@yc}}
  \pgfpathlineto{\pgfpoint{\pgf@xc}{\pgf@yc}}
  }
}
\makeatother

\makeatletter
\pgfdeclareshape{stickyback}{
  \inheritsavedanchors[from=rectangle] % this is nearly a rectangle
  \inheritanchorborder[from=rectangle]
  \inheritanchor[from=rectangle]{center}
  \inheritanchor[from=rectangle]{north}
  \inheritanchor[from=rectangle]{south}
  \inheritanchor[from=rectangle]{west}
  \inheritanchor[from=rectangle]{east}
  % ... and possibly more
  \backgroundpath{% this is new
  % store lower right in xa/ya and upper right in xb/yb
  \southwest \pgf@xa=\pgf@x \pgf@ya=\pgf@y
  \northeast \pgf@xb=\pgf@x \pgf@yb=\pgf@y
  % compute corner of ‘‘flipped page’’
  \pgf@xc=\pgf@xa \advance\pgf@xc by+20pt % this should be a parameter
  \pgf@yc=\pgf@ya \advance\pgf@yc by+20pt
  % construct main path
  \pgfpathmoveto{\pgfpoint{\pgf@xa}{\pgf@yc}}
  \pgfpathlineto{\pgfpoint{\pgf@xa}{\pgf@yb}}
  \pgfpathlineto{\pgfpoint{\pgf@xb}{\pgf@yb}}
  \pgfpathlineto{\pgfpoint{\pgf@xb}{\pgf@ya}}
  \pgfpathlineto{\pgfpoint{\pgf@xc}{\pgf@ya}}
  \pgfpathclose
  }
}
\makeatother

\NewDocumentCommand\StickyNoteFront{O{6cm}mO{6cm}}{%
\begin{tikzpicture}
\node[
stickyfront,
draw,
inner xsep=7pt,
fill=myyellow,
inner ysep=10pt
] {\parbox[t][#1][c]{#3}{#2}};
\end{tikzpicture}%
}

\NewDocumentCommand\StickyNoteBack{O{6cm}mO{6cm}}{%
\begin{tikzpicture}
\node[
stickyback,
draw,
inner xsep=7pt,
fill=myyellow,
inner ysep=10pt
] {\parbox[t][#1][c]{#3}{#2}};
\end{tikzpicture}%
}

% TF-IDF math operators
\DeclareMathOperator{\tfidf}{tf-idf}
\DeclareMathOperator{\tf}{tf}
\DeclareMathOperator{\idf}{idf}
% Argmax
\DeclareMathOperator*{\argmax}{\arg\!\max}

% Hashtag format
\newcommand{\hashtag}[1]{{``\##1''}}
\section{Obtaining a dataset}
There is currently a lack of publicly available datasets for group recommendation.

We have several requirements to a dataset. The bigger the dataset the better. Datasets such as the HappyMovie dataset is,regrettably, too small, but presents many of the necessary traits of a good group recommender dataset.

It needs users and groups. There are several options to go with here. The groups could be ephemeral or consist of friends or anything in between, but it has an effect on gathering data, as you unconsciously affect the data. A truly random selection of people for a group could introduce obstacles uncommon for a group of friends, such as sharing no language common to all for a movie recommendation, which could become the main challenge of a good recommendation to the exclusion of individual taste for such a configuration. Such a problem would not be addressable depending on how the groups are formed.

Having many groups of various sizes would be preferable, as group recommender systems perform differently on various group sizes.

Assuming a new dataset will not become available, there are three ways for us to make our own. However, the necessary scale in order to get a dataset of the right size makes it a difficult task to do in an inexpensive manner.

The hardest is to create or contact a popular service provider in the domain of group recommendation, and siphon off data of user behavior over time. This is similar to the method used by Netflix for their dataset.
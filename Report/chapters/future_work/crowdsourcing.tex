\section{Obtaining a dataset}
There is currently a lack of publicly available datasets for group recommendation.

We have requirements as to what makes an ideal dataset for our problem domain. The bigger the dataset the better for accurate learning and for splitting the data into subsets for validation. Datasets such as the HappyMovie dataset \cite{HappyMovie} is too small, but presents many of the necessary traits of a good group recommender dataset.

It needs users and groups. There are several options to go with here. The groups could be ephemeral or consist of friends or anything in between, but it has an effect on gathering data, as you unconsciously affect the data. A truly random selection of people for a group could introduce obstacles uncommon for a group of friends, such as sharing no language common to all for a movie recommendation, which could become the main challenge of a good recommendation to the exclusion of individual taste for such a configuration. Such a problem would not be addressable depending on how the groups are formed. Having many groups of various sizes would be preferable, as group recommender systems perform differently on various group sizes. It is not a problem whether users shows up in more than one group. It is a good cost-saving measure to get more data out of fewer people. The data should have ratings of items in such a way that individual preferences can be inferred. However, the most important part are the group recommendations. It would be ideal to have groups of users make an individual ranked list each  and one for the group together. This could be used to both measure the satisfaction of the users and also give us an idea of how well we reflect the decision making process.

Assuming a new dataset will not become available, there are two ways for us to make our own. However, the necessary scale in order to get a dataset of the right size makes it a difficult task to do in an inexpensive manner.

The hardest option is creating or convincing a popular service provider in the domain of group recommendation to help provide data of user behavior. Netflix is an example of a service provider who leveraged their position to create a dataset. A survey could also be launched and spread through social media, with the trade between survey and user being behavioral data in return for useful insights about habits or traits, but it relies on the survey going viral, which is an unreliable path.

The alternative is gathering the data for the sole purpose of making the dataset, which is the most straightforward. We have two options for approaching this. Find willing participants for a survey, and a crowd sourcing.

Finding enough willing participants is main problem, as the dataset needs at the lowest enough to form several hundred groups. Though there are more options than ever through services like Mechanical Turk by Amazon, crowd sourcing presents its own challenges. Participants here are paid for their time and effort in the order of 10 to 50 cents per user, which is a reliable way of motivating participation, but makes a large dataset expensive. Another challenge is introducing the group decision aspect, as known crowd sourcing sites do not have the framework for groups of workers to cooperate on a task simultaneously, rather than independently, and reach a consensus before submitting results. There are possible ways to offset this, such as simulating group recommendations and survey the user on their satisfaction with the results, but it definitely requires careful consideration of how to proceed in order to get useful data.

%Homepage where people give their preferences and form groups for a short test as a tool for gathering data
\subsubsection{SVD++} \label{bg:mf_svd++}
As the name implies, SVD++ is an extension on SVD. Where SVD in general only works with explicit data, SVD++ takes into account the implicitly gathered data as well. Take for instance a user that has a number of items they have shown some indirect interest in, for instance hovering over the items to see some information about them, SVD++ would specifically account for this set of potentially interesting items\cite{svd++}.

The standard model for SVD can be seen in Equation \ref{eq:svd_R}. On could modify on this in order to only find on rating which would look like Equation \ref{eq:svd++_one_rating}.

\begin{equation} \label{eq:svd++_one_rating}
\centering
\hat{R}_{ui} = \sum^f_{f=1}(A_{uf}B_{fi})
\end{equation}

In SVD++ the last part of the model is altered, to reflect how users are characterized based on the items they have rated, as seen in Equation \ref{eq:svd++_standard}. Here $R(u)$ is the set of items rated by the user $u$, and $y_{j}$ is a factor vector. $A_{u}$ is a free user-factors vector of size $f$ equal to the number of latent features.

\begin{equation} \label{eq:svd++_standard}
\centering
\hat{R}_{ui} =\sum^f_{f=1}\left(B_{fi}\left (A_{uf} + \left | R(u) \right |^{-\frac{1}{2}} \sum_{j\in R(u)} y_{j} \right )\right)
\end{equation}

There is in theory no limit to the number of itemsets that can be inferred and accounted for, as each would simply have its own item factor vector, as seen in Equation \ref{eq:svd++_multi}.

\begin{equation} \label{eq:svd++_multi}
\centering
\hat{R}_{ui} = \sum^f_{f=1}\left( B_{fi}\left (A_{uf} + \left | N^{1}(u) \right |^{-\frac{1}{2}} \sum_{j\in N^{1}(u)} y_{j}^{(1)} + \left | N^{2}(u) \right |^{-\frac{1}{2}} \sum_{j\in N^{2}(u)} y_{j}^{(2)} \right )\right)
\end{equation}

SVD++ in general performs better than SVD, however if there is no implicit data, the advantage in using SVD++ becomes negligible.
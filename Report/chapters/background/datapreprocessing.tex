\section{Data Preprocessing}
When working with data and datasets, it is important to ensure that the data format is structured in a way that is usable by the algorithms being implemented or tested. To this end data preprocessing is an obvious part of the solution. Typically this consists of formatting, filtering, and splitting the data.

As the data is not always on the correct format, it is sometimes necessary to perform some sort of formatting on it, for instance transforming a tab-separated file into a proper matrix-format for use in matrix related calculations.

Filtering or cleaning is simply the art of cleaning the dataset of irrelevant data, such as the age and gender of users in an experiment, where who and what the users are play no role. To some extend it is also used to correct data entries that would be corrupted or otherwise unusable, for instance uninterpretable entries being replaced by either what was supposed to be there or a non-influential value.

Lastly the data is often separated into equally sized parts in preparation for validation methods such as k-fold testing.
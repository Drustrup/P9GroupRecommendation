\section{Single Transferable Vote} \label{sec:stv}
Aggregations are not the only way to reach a consensus as a group, another prominent method is voting or electoral systems. We will only present one of the many systems here, namely single transferable vote(STV), a system which have seen use for more than 90 years\cite{stvireland}.
STV is speculated to be one of the best electoral systems, in the sense that it seeks to reduce the number of wasted votes. The system operates on ranked lists of candidates, which can directly be translated into a ranked list for a ranked aggregation, making it relevant to take into account for our approach.

The basis of the system is having a number of open positions, also called seats; a number of candidates for the positions; the total number ranked lists with the votes; and a threshold that each candidate have to pass to guarantee their seat. The threshold can be simply calculated as the total votes divided by the number of seats\cite{stv}.

Following the flow depicted in Figure \ref{fig:stvflow}, firstly the votes for each candidate are summed up and if any candidates exceed the threshold the excess votes, is transfered to other candidates based on the other candidates in the voting lists, typically based on the proportions, for instance candidate A has reached the threshold and amongst his voters 30\% prefers candidate B as their second choice, 60\% prefers candidate C, and the remaining have no secondary preference. Therefore, the excess votes should be assigned according to these proportions, while the 10\% that had no secondary preference would be discarded which results in a slight loss of votes.

Afterwards, the candidate with the currently lowest amount of votes will be eliminated and their votes partitioned out similarly to before. This is repeated until all the seats have been filled, however because votes are occasionally tossed out, the procedure should also stop when the number of candidates has been reduced to be equal to the number of seats. The method does not inherently handle cases where there are more seats than candidates.

Based on how the system functions, it is not directly applicable to the problem of selecting a satisfying set of items for the scenario we work with. This is because having only one vote for each user would, in cases where users are without many similarities and with the $k$ in top-k being close to the number of group members, cause the system to select only the highest rated item for each user, as the threshold would be close to 1. If the size of the group is large, the system could still work, but in realistic scenarios sizes big enough to matter will not occur. However using STV as a foundation new approaches can be developed, which will be discussed in Section \ref{BTC}.

\begin{figure}
	\centering
	\includegraphics[scale=0.5]{STV_flow.png}
	\caption{Flow of an STV voting process}
	\label{fig:stvflow}
\end{figure}
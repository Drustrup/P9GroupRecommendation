\section{Problem Areas}
\todo{Describe the discussed problems fund within the group recommendation area}

Within the group recommendation field, there are a few areas where there will typically be problems. Some of the problems carry over from recommender systems for individuals, for instance the cold-start problem and the popularity bias. Others are more specific to group recommending, for instance how to replicate group decision making and the satisfaction, while one is caused by the need for automatic verification of the recommending results. These problems are described more in-depth here. \note{remove last sentence?}

\subsection{Cold-start}
The cold-start problem is well known within the field of recommender systems, and it is also present in group recommendation. Cold-start refers to issue that a new user in most cases do not have much data to recommend based on. In group recommenders the problem can be said to be less prevalent as a new user joining a group can, from the systems perspective, act as a passive member of the group until enough data has been gathered. However there will still be problems when a large number of users in a group are new, as then the system does not work for them. It is a common problem and solutions to it that are applicable to recommender systems in general will likely be usable in group recommending as well.

\subsection{Popularity Bias}
A common problem found in many recommender systems is the bias towards popular items. Whether this is a problem or not depends on the goal of the recommender, for instance if the goal is to recommend new music to an individual, it might not be the best choice to recommend songs from the current top 100 hit-list as they have likely already heard them. 

\subsection{Group Decisions and Satisfaction}

\subsection{Automatic Verification}
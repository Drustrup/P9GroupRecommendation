\chapter{Introduction}
Recommender systems are a natural response to the problem of information overload.

With a lot of information available a task does not necessarily get easier. When people need to make a choice when faced with complex or abundant information, they simplify it in order to better make sense of the information presented and hasten decision making at the cost of not using all the information available. Recommender systems are a tool to solve that same common challenge for various domains. In philosophy, Thomas Hobbes and others argued that a social contract exists, where people are willing to submit to the will of an outer authority out of self-interest and rationality, whether that authority is a dictator or elected assembly\footnote{For the Social Contract theory, we refer to http://www.iep.utm.edu/soc-cont/ and https://en.wikipedia.org/wiki/Social\_contract}. They both serve the same purpose of making a choice on their behalf.

Today, machine learning helps making informed choices sometimes on behalf of humans for their convenience, trading stocks or working various machine learning tasks. Humans are willingly trading decision power in return for convenience. For a group of people putting aside their individual preferences in order to choose one among several items exists an unspoken social contract between them to accept the result of the process used to reach consensus. A group recommender could be the core of this process, as it could reach a decision in the time it takes humans to pose the question among various other benefits. This motivates the need for a good recommender system.

Group recommender systems change the game and introduce new problems in the area of agreement. To avoid information overload, a group can defer to an outside authority to ease decision making, however doing so transfers some of the problems the group had in agreeing to the system. For a single user, the best choice in a recommendation is what will satisfy the user. However, add more people and disagreement often follows, as some people's preferences are neglected. Kenneth J. Arrow proved this point back in 1948, that no aggregation method could possibly ensure everyone was satisfied\cite{arrow}.

This results in that group recommendations premiere problem being reconciling the differing opinions in the group in a manner that is satisfactory for the group.

%However most of the research in the area has been for single user recommender systems, rather than for groups. The reason for this can be found in the evolution of the internet as a single user experience, which has lead to single user services, such as movie streaming services such as Netflix and Youtube, which are built around the idea of one person accessing their site at a time. The needs of such companies for powerful recommendation systems have pushed the envelope for recommender systems.
%Throughout time recommendations have often been a determined factor in decision making.
%The first recommendations were word-of-mouth recommendations where family and friends would proclaim their opinion about certain products like movies, books, or traveling destinations. 

%In the past decades the way we receive these recommendations have changed a lot. Today recommendations often occur online and there exist a recommender for almost every item or service available. The difference between these recommenders compared to word-of-mouth is that they are based on either what other people similar to you like or based on items similar to those associated with you, respectively collaborative or content-based recommendations.

%Even though recommenders is a big part of our everyday lives, there are some situations where there is a shortage in the recommender field. One of these shortcomings is group recommendation, like having a group of people deciding where to go on vacation, which movie to watch, or what music to hear. Some research have been done in the filed, but no method have yet been able to reflect the decision making process of a group. In this report we look into the problem of recommending to groups in order to obtain an acceptable level of satisfaction in the group.
\section{Related Work}
\input{chapters/introduction/problem_areas}
\section{Problem Statement}
The purpose of recommending differs depending on the type of recommender. Looking at music recommendation for individuals, the main purpose of the recommendation is to recommend new music the individual would like and at the same time not only recommend from the popularity bias. This purpose changes if we want to recommend music to groups, where a group by our definition is 4 people or more. Instead of finding new music, the focus is on satisfying every person in the group to some extend, which does not necessarily require avoiding the popularity bias. 

This though raises the question of how to measure the satisfaction level in a group of people based on the satisfaction of each persons opinion of the music played. This leads us to the question: 
\textit{How can recommendations be made to a group of people by reflecting a groups decision making process, to ensuring a high level of satisfaction in the group?} This statement opens for questions like:
\begin{itemize}
\item How to make recommendations to a group of people based on the individual users preferences?
\item How to measured the satisfaction level in a group in regards to the music recommended?
\item How to explain the reasoning for the recommendation made to the group? \note{Maybe just focus on the first two?}
\end{itemize}
\section{Scenarios}

\subsection{Group Movie watching}
Size: 3-6
Group needs to decide on a movie to watch. They are able discuss it among themselves. The people do not necessarily know each other, but they are likely friends or acquaintances.
\subsection{Gathering with Music}
size: 3-6
Group needs to decide on the sequence of music playing. There are few enough that a sort of consensus could be reached via dialog. All within hearing distance of each other. The people do not necessarily know each other, but they are likely friends or acquaintances.
\subsection{Party with Music}
size: 20-100
Group needs to decide on the sequence of music playing. They are able to communicate, however from the size and context of the event, it is impractical as the focus is elsewhere for most people to quickly reach a consensus. There are likely clusters of friends, however the group is more randomly put together.



%Start with hobbe's contract theory to argue for people giving up their freedom by choice to trust in laws and organizations
%Lead into history of recommendation and Kenneth J. Arrow
%Aggregation is imperfect - use as a quote
%as we don't have a ground truth, we'll need a new way to verify our recommender
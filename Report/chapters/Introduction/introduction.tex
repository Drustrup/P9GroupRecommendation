\chapter{Introduction}
Recommender systems are a natural response to the problem of information overload.

With a lot of information available a task does not necessarily get easier. When we need to make a choice when faced with complex information, we simplify it in order to better make sense of them and hasten decision-making. Recommender systems are a tool to solve that same common challenge for various domains. In philosophy, Thomas Hobbes argued that a social contract exists where people are willing to submit to the will of an outer authority out of self-interest and rationality, whether that authority is a dictator or elected assembly. They both serve the same purpose of making a choice on his behalf.\cite{Hobbes}

Today, machine learning makes informed choices on behalf of humans for their convenience, trading stocks or working various machine learning tasks. Humans are willingly trading decision power in return for the convenience. For a group of people putting aside their individual preferences in order to choose one among several items exists an unspoken social contract between them to accept the result of the process used to reach consensus. A group recommender could be the mainstay of this process, as it could reach a decision in the time it takes humans to pose the question. This motivates the need for a good recommender system.

Group recommender systems change the game and introduce new problems in the area of agreement. To avoid information overload, a group can defer to an outside authority to ease decision making, however doing so transfers problems from politics and ethics that are not easily answered. For a single user, the best choice in a recommendation is what will satisfy the user. However add more people and you get disagreement, meaning that some people's opinions are ignored. Kenneth J. Arrow proved this point back in 1948, that no aggregation method could possible ensure everyone was happy.

This ends out in that group recommendation's premiere problem being reconciling the differing opinions in the group in a manner that is both ethical and satisfactory for the group.

%However most of the research in the area has been for single user recommender systems, rather than for groups. The reason for this can be found in the evolution of the internet as a single user experience, which has lead to single user services, such as movie streaming services such as Netflix and Youtube, which are built around the idea of one person accessing their site at a time. The needs of such companies for powerful recommendation systems have pushed the envelope for recommender systems.
%Throughout time recommendations have often been a determined factor in decision making.
%The first recommendations were word-of-mouth recommendations where family and friends would proclaim their opinion about certain products like movies, books, or traveling destinations. 

%In the past decades the way we receive these recommendations have changed a lot. Today recommendations often occur online and there exist a recommender for almost every item or service available. The difference between these recommenders compared to word-of-mouth is that they are based on either what other people similar to you like or based on items similar to those associated with you, respectively collaborative or content-based recommendations.

%Even though recommenders is a big part of our everyday lives, there are some situations where there is a shortage in the recommender field. One of these shortcomings is group recommendation, like having a group of people deciding where to go on vacation, which movie to watch, or what music to hear. Some research have been done in the filed, but no method have yet been able to reflect the decision making process of a group. In this report we look into the problem of recommending to groups in order to obtain an acceptable level of satisfaction in the group.
\input{chapters/introduction/related_work}
\input{chapters/introduction/problem_areas}
\section{Problem Statement}
The purpose of recommending differs depending on the type of recommender. Looking at music recommendation for individuals, the main purpose of the recommendation is to recommend new music the individual would like and at the same time not only recommend from the popularity bias. This purpose changes if we want to recommend music to groups, where a group by our definition is 5 people or more. Instead of finding new music, the focus is on satisfying every person in the group to some extend, which does not necessarily require avoiding the popularity bias. 

This though raises the question of how to measure the satisfaction level in a group of people based on the satisfaction of each persons opinion of the music played. This leads us to hed question: 
\textit{How can recommendations be made to a group of people by reflecting a groups decision making process, to ensuring a high level of satisfaction in the group?} This statement opens for questions like:
\begin{itemize}
\item How to make recommendations to a group of people based on the individual users preferences?
\item How to measured the satisfaction level in a group in regards to the music recommended?
\item How to explain the reasoning for the recommendation made to the group?
\end{itemize}
\section{Scenario}
%Initially there were three kinds of scenarios that we believed would benefit from a group recommender. A single item recommendation to a small group of friends or acquaintances, for instance which movie to watch. A multi item recommendation to a small group of friends or acquaintances, for instance background music for a small get-together. And lastly a multi item recommendation for a larger group of mostly unacquainted people, for instance at a pub. The first two scenarios could be solved by the group members amongst themselves much more easily than the last one. This means that a group recommender would be much more beneficial to the last scenario.\\

For the purposes of this project, the scenario we will be working on is defined as selecting the music to be played in a public area or online gathering. The group size will be between 4 or more people. The composition of the group will reflect many small cliques with few connections between communities, as each group member might be acquainted with some other member, but unlikely to know everyone, as the members of the group are seen as random. We do not concern ourselves with the nationality, or background of each group member, as while the group members are spatially bound to the same location when meeting in a public place, this is not the case for an online meeting. With virtual reality becoming more prominent in recent years, a gathering could easily consist of people from all over the world meeting with nothing more in common than ownership of a virtual reality device and an Internet connection. This makes it interesting to present a general idea of finding the best music for an impromptu gathering of random individuals. %https://techcrunch.com/2016/05/08/vr-party-is-a-social-dj-dance-club-for-virtual-reality/

%Motivate the group recommender for this scenario
The group recommender is necessary in this scenario because sorting through the preferences of even 4 people is an exhausting task, worse than the original motivating factor of information overload for a single person by magnitudes. By removing the people from the decision making process and automating it, not only are the group members free to do other things, but the choice of recommendation does not rely on the initiative of the group, which would lead to no selection being made or a few individuals taking charge in a dictatorial fashion.

To summarize, for a group, $M$, ranging from diverse to homogeneous of random members, each with $T$ tracks listing how each user rates each track, we will find the track such that everyone are as satisfied as can be. With the goal of maximizing satisfaction of the members, $S$, overall, the group's satisfaction is a product of each individual user's satisfaction with a track, $s_mt$, as shown in \ref{eq:satisfaction}, with $f$ being an function that aggregates over all the satisfaction values.

\begin{equation}\label{eq:satisfaction}
	\centering
	\max_{t \in T} S_t = \sum_{m=1}^{M} f(s_{mt})
\end{equation}

%There are no demographic differences noted, such as age or gender, instead the members are defined solely by the music they listen to.
%The data for each member will be gathered arbitrarily from a lastfm dataset containing users and their listening habits\todo{Find a way to be better than average or random}. For the sake of simplicity the group is seen as of static size \note{We can discuss and change this statement later}, once it has been created it will not change. The satisfaction of the group will ideally be measured by increasing the income of the local. Another measure would be increasing the number of patrons, but regardless of which measure is used, the recommender should aim to not make it worse. \todo{Add how we will verify/test}.



%Start with hobbe's contract theory to argue for people giving up their freedom by choice to trust in laws and organizations
%Lead into history of recommendation and Kenneth J. Arrow
%Aggregation is imperfect - use as a quote
%as we don't have a ground truth, we'll need a new way to verify our recommender
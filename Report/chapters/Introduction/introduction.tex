\chapter{Introduction} 
Throughout time recommendations have often been a determined factor in decision making.
The first recommendations were word-of-mouth recommendations where family and friends would proclaim their opinion about certain products like movies, books, or travelling destinations. 

In the past decades the way we receive these recommendations have changed a lot. Today recommendations often occur online and there exist a recommender for almost every item or service available. The difference between these recommenders compared to word-of-mouth is that they are based on either what other people similar to you like or based on items similar to those associated with you, respectively collaborative or content-based recommendations.

Even though recommenders is a big part of our everyday lives, there are some situations where there is a shortage in the recommender field. On of these shortcomings is group recommendation, like having a group of people deciding where to go on vacation, which movie to watch, or what music to hear. Some research have been done in the filed, but no method have yet been able to reflect the decision making process of a group. In this report we look into the problem of recommending to groups in order to obtain an acceptable level of satisfaction in the group.
\section{Related Work}
\input{chapters/introduction/problem_areas}
\section{Problem Statement}
The purpose of recommending differs depending on the type of recommender. Looking at music recommendation for individuals, the main purpose of the recommendation is to recommend new music the individual would like and at the same time not only recommend from the popularity bias. This purpose changes if we want to recommend music to groups, where a group by our definition is 4 people or more. Instead of finding new music, the focus is on satisfying every person in the group to some extend, which does not necessarily require avoiding the popularity bias. 

This though raises the question of how to measure the satisfaction level in a group of people based on the satisfaction of each persons opinion of the music played. This leads us to the question: 
\textit{How can recommendations be made to a group of people by reflecting a groups decision making process, to ensuring a high level of satisfaction in the group?} This statement opens for questions like:
\begin{itemize}
\item How to make recommendations to a group of people based on the individual users preferences?
\item How to measured the satisfaction level in a group in regards to the music recommended?
\item How to explain the reasoning for the recommendation made to the group? \note{Maybe just focus on the first two?}
\end{itemize}
\section{Scenarios}

\subsection{Group Movie watching}
Size: 3-6
Group needs to decide on a movie to watch. They are able discuss it among themselves. The people do not necessarily know each other, but they are likely friends or acquaintances.
\subsection{Gathering with Music}
size: 3-6
Group needs to decide on the sequence of music playing. There are few enough that a sort of consensus could be reached via dialog. All within hearing distance of each other. The people do not necessarily know each other, but they are likely friends or acquaintances.
\subsection{Party with Music}
size: 20-100
Group needs to decide on the sequence of music playing. They are able to communicate, however from the size and context of the event, it is impractical as the focus is elsewhere for most people to quickly reach a consensus. There are likely clusters of friends, however the group is more randomly put together.
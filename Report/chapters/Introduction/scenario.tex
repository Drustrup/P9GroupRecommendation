\section{Scenario}\label{sec:introduction_scenario}
%Initially there were three kinds of scenarios that we believed would benefit from a group recommender. A single item recommendation to a small group of friends or acquaintances, for instance which movie to watch. A multi item recommendation to a small group of friends or acquaintances, for instance background music for a small get-together. And lastly a multi item recommendation for a larger group of mostly unacquainted people, for instance at a pub. The first two scenarios could be solved by the group members amongst themselves much more easily than the last one. This means that a group recommender would be much more beneficial to the last scenario.\\

For the purposes of this project, the scenario we will be working on is defined as selecting the music to be played in a public area or online gathering. The group size will be 4 or more people. The composition of the group will reflect many small cliques with few connections between communities, as each group member might be acquainted with some other member, but unlikely to know everyone, as the members of the group are seen as random. We do not concern ourselves with the nationality, or background of each group member, as while the group members are spatially bound to the same location when meeting in a public place, this is not the case for an online meeting. With virtual reality becoming more prominent in recent years, a gathering could easily consist of people from all over the world meeting up online in a virtual gathering. This makes it interesting to present a general idea of finding the best music for an impromptu gathering of random individuals. %https://techcrunch.com/2016/05/08/vr-party-is-a-social-dj-dance-club-for-virtual-reality/

%Motivate the group recommender for this scenario
The group recommender is necessary in this scenario, because sorting through the preferences of even 4 people is an exhausting task, worse than the original motivating factor of information overload for a single person by magnitudes. By removing the people from the decision making process and automating it, not only are the group members free to do other things, but the choice of recommendation does not rely on the initiative of the group, which would lead to no selection being made or a few individuals taking charge in a dictatorial fashion.

%To summarize, for a group, $M$, ranging from diverse to homogeneous of random members, each with $T$ tracks listing how each user rates each track, we will find the track such that everyone are as satisfied as can be. With the goal of maximizing satisfaction of the members, $S$, overall, the group's satisfaction is a product of each individual user's satisfaction with a track, $s_mt$, as shown in \ref{eq:satisfaction}, with $f$ being an function that aggregates over all the satisfaction values.

%\begin{equation}\label{eq:satisfaction}
%	\centering
%	\max_{t \in T} S_t = \sum_{m=1}^{M} f(s_{mt})
%\end{equation}

%There are no demographic differences noted, such as age or gender, instead the members are defined solely by the music they listen to.
%The data for each member will be gathered arbitrarily from a lastfm dataset containing users and their listening habits\todo{Find a way to be better than average or random}. For the sake of simplicity the group is seen as of static size \note{We can discuss and change this statement later}, once it has been created it will not change. The satisfaction of the group will ideally be measured by increasing the income of the local. Another measure would be increasing the number of patrons, but regardless of which measure is used, the recommender should aim to not make it worse. \todo{Add how we will verify/test}.
\section{Problem Statement}
The purpose of recommending differs depending on the type of recommender. Looking at music recommendation for individuals, the main purpose of the recommendation is to recommend new music the individual would like and at the same time not only recommend from the popularity bias. This purpose changes if we want to recommend music to groups, where a group by our definition is 5 people or more. Instead of finding new music, the focus is on satisfying every person in the group to some extend, which does not necessarily require avoiding the popularity bias. 

This though raises the question of how to measure the satisfaction level in a group of people based on the satisfaction of each persons opinion of the music played. This leads us to hed question: 
\textit{How can recommendations be made to a group of people by reflecting a groups decision making process, to ensuring a high level of satisfaction in the group?} This statement opens for questions like:
\begin{itemize}
\item How to make recommendations to a group of people based on the individual users preferences?
\item How to measured the satisfaction level in a group in regards to the music recommended?
\item How to explain the reasoning for the recommendation made to the group? \note{Maybe just focus on the first two?}
\end{itemize}
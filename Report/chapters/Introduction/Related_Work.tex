\section{Related Work} \label{in:related_work}

In the literature a multitude of approaches to group recommendation have been proposed and pursued. As there is too many approaches to cover in full, the focus will be held on a few areas, namely various aspects of Matrix Factorization in correlation to group recommendation and possible uses of Nash Equilibrium in terms of determining the satisfaction of the group.

While group recommendation has typically been done using a variety of aggregation strategies, recent research has explored the possibility of using Matrix Factorization based Collaborative Filtering to handle the recommendations \cite{mfbasedcf}. Their approach was shown to be 25-50\% better than the existing standard giving valid reasoning to the idea that Matrix Factorization is an appropriate tool to use in regards to recommender systems.

Another approach explored in "User’s Satisfaction in Recommendation Systems for Groups: an Approach Based on Noncooperative Games"\cite{nashequilibrium} looked at Nash Equilibrium as a potential solution to the satisfaction problem. Their results showed that Nash Equilibrium performed slightly worse than the Average aggregation strategy, but that increases in group size reduced this gap.

A look at the idea that a group is in competition with itself is interesting and intuitively sound, but if one looks at how the idea works out in practice, such as an election system that encourages strategic voting, you end up with an overall reduced satisfaction. This can be explored with the prisoner's dilemma, which is a well-known thought exercise in game theory. The setup is that two prisoners can either stay silent about their crime or tell on the other prisoner. If they both confess, they incur a cost, however if both lie they incur only a minor cost. Though if one confesses to the crimes, but the other attempt to lie to escape the crime, then the liar will incur the lion's share of the cost. Thus, the winning strategy for both prisoners in game theory is to confess. However, the option where both prisoners could get off with only a slap on the wrist if both lied is still the best possible outcome for the group as a whole, however in game theory that option is bypassed. This shows us that attempting to cater to the whims of single individuals should effectively hurt the satisfaction of the entire group.
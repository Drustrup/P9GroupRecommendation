\section{Related Work} \label{in:related_work}

In the literature a multitude of approaches to group recommendation have been proposed and pursued. As there is too many approaches to cover in full, the focus will be held on a few areas, namely various aspects of Matrix Factorization in correlation to group recommendation and possible uses of Nash Equilibrium in terms of determining the satisfaction of the group.

While group recommendation has typically been done using a variety of aggregation strategies, recent research has explored the possibility of using Matrix Factorization based Collaborative Filtering to handle the recommendations \cite{mfbasedcf}. Their approach was shown to be 25-50\% better than the existing standard giving valid reasoning to the idea that Matrix Factorization is an appropriate tool to use in regards to recommender systems.

Another approach explored in "User’s Satisfaction in Recommendation Systems for Groups: an Approach Based on Noncooperative Games"\cite{nashequilibrium} looked at Nash Equilibrium as a potential solution to the satisfaction problem. Their results showed that Nash Equilibrium performed slightly worse than the Average aggregation strategy, but that increases in group size reduced this gap.
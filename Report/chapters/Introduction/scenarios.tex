\section{Scenarios}
Initially there was three kinds of scenarios that we believed would benefit from a group recommender. A single item recommendation to a small group of friends or acquaintances, for instance which movie to watch. A multi item recommendation to a small group of friends or acquaintances, for instance background music for a small get-together. And lastly a multi item recommendation for a larger group of mostly unacquainted people, for instance at a pub. The first two scenarios could be solved by the group members amongst themselves much more easily than the last one. This means that a group recommender would be much more beneficial to the last scenario.\\

For the purposes of this project, the scenario we will be working on is define as selecting the music to be played at a pub. The size of the group will be 15-100 people, depends on things like time of day and capacity of the local. The members of the group are seen as random, as such all members can be seen as equal. There is no demographic differences, such as age or gender, instead the members are defined solely by the music they listen to. The data for each member will be gathered through God's divine blessing \todo{Find a way to be better than God}. For the sake of simplicity the group is seen as of static size \note{We can discuss and change this statement later}, once it has been created it will not change. The satisfaction of the group will ideally be measured by increasing the income of the local. Another measure would be increasing the number of patrons, but regardless of which measure is used, the recommender should aim to not make it worse. As these measurements are hard to test for, we will be using the Devil's beliefs to compare to \todo{Make a way to forsake the Devil}.
\section{Individual Recommendation}\label{sec:individual_recommendation}
In Section \ref{sec:decision} we determined that SVD would be a suitable method of recommendation for this project. 
We have experimented with two different approaches with slight differences in how they do regularization and in how the A and B matrices are constructed.

The first approach was the Funk-SVD which can be seen in Equation \ref{eq:indi_rec_funk}, where Equation \ref{eq:indi_rec_funk_error} defines the error\cite{svdsimonfunk}. In the post Simon Funk mentions that the algortihm uses a method somewhat like Tikhonov regularization\footnote{https://en.wikipedia.org/wiki/Tikhonov_regularization}, denoted as $\lambda$ in the equation. We have made some adjustments compared to Funks documentation\cite{svdsimonfunk}. Instead of calculating the values of matrix A and B based on average movie rating and average offset for users, our initial test showed the best performance when populating them with random variables between 0 and 1.
\begin{equation}\label{eq:indi_rec_funk_error}
error = R_{ui} - \sum_n A_{un}B_{ni}
\end{equation}

\begin{equation}\label{eq:indi_rec_funk}
\begin{aligned}
A_{uk} += \eta * (error * B_{ki} - \lambda A_{uk}) \\
B_{ki} += \eta * (error * A_{uk} - \lambda B_{ki})
\end{aligned}
\end{equation}

The second approach can be seen in Equation \ref{slides}, which is also described in Section \ref{bg:sub:factorizationmodels}. As mentioned this approach has some differences compared to Funk-SVD. First of all, it uses a simple weighted-decay for regularization and not Tikhonov. Second, the A and B matrices for this approach is populated using the SVDS method in MatLab, which is similar to Equation \ref{eq:svd_AB_squeezing} in Section \ref{bg:sub:factorizationmodels}.

\begin{equation}\label{slides}
\begin{aligned}
A_{uk} += \eta * error *B_{ki} - \lambda A_{uk} \\
B_{ki} += \eta * error *A_{uk} - \lambda B_{ki} 
\end{aligned}
\end{equation}

In Listing \ref{lst:SVD}, it is shown how we trained both of the algorithms. We have a while-loop that runs through a specified number of iterations or as long as the RMSE value does not grow too much.
We train both methods using stochastic gradient descent so we select a random known rating to fit our A and B matrices against. Then, starting on Line \ref{line:modulo}, for every 1000 iterations we calculate the RMSE to ascertain that the error is decreasing. 

\begin{lstlisting}[caption={Training of the prediction matrix},label=lst:SVD,escapechar=|]
...
While n < threshold && RMSE < thresholdRMSE
	x = randomKnownRatingCoordinates
	
	SVD-algorithm(x)
	
	if modulo(threshold,1000) == 0 |\label{line:modulo}|
		predictionMatrix = A * B
		predictionArray = extractValidationPredictions(predictionRating)
		RMSE = sqrt(MSE(predictonArray,validationArray)	
		
		if RMSE < lowRMSE 
			lowRMSE = RMSE 
			thresholdRMSE = RMSE + 0.1
		end
	end
	n += 1
end
\end{lstlisting}

During training we needed to tune the learning rate $\eta$, regularization value $\lambda$, and the number of features $f$ for each algorithm. A selection of these tests can be seen in Table \ref{fig:funk_tune} and Table \ref{fig:SVD_tune}.
\begin{table}[h]
\centering
\begin{minipage}{.48\textwidth}\centering
\begin{tabular}{|l|llll|}
\hline
Setup	& $\eta$ & $\lambda$ & f   & RMSE   \\ \hline
$SA_1$	& 0.001  & 0.01      & 100 & 1.838 \\ \hline
$SA_2$	& 0.01   & 0.01      & 100 & 1.283 \\ \hline
$SA_3$	& 0.01   & 0.2       & 100 & 1.029 \\ \hline
$SA_4$	& 0.01   & 0.2		 & 50  & 0.994 \\ \hline
\end{tabular}
\captionof{table}{Parameter tuning of Funk-SVD}
\label{fig:funk_tune}
\end{minipage}
\hfill
\begin{minipage}{.48\textwidth}\centering
\begin{tabular}{|l|llll|}
\hline
Setup	& $\eta$ & $\lambda$ & f   & RMSE \\ \hline
$SB_1$	& 0.01   & 0.01   & 100 & 1.387   \\ \hline
$SB_2$	& 0.05   & 0.01   & 100 & 1.006  \\ \hline
$SB_3$	& 0.01   & 0.2    & 10 	& 1		  \\ \hline
\end{tabular}
\captionof{table}{Parameter tuning of SVD}
\label{fig:SVD_tune}
\end{minipage}
\end{table}

We decided that a RMSE below 1 would be sufficient as the individual recommendation is not the main focus and we would not use too much time on parameter tuning. Based on the different setups we ended up using the Funk-SVD with the parameters in setup $SA_4$ from Table \ref{fig:funk_tune}.
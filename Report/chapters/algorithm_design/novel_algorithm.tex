\section{Novel Aggregation Emphasizing Reasoning}
\todo{Change name of section, preferably to something with a catchy acronym}
Encompassing the findings in Section \ref{bg:aggregation:groupbehavior}, we construct our own novel approach to the selection process in a group recommender. The approach is intended to put emphasis on the ratings from group members that like the tracks the most. This will be accomplished by using the multiplicative aggregation strategy on a small number of the highest ratings , while the remaining ratings are handled with the average strategy.

With this approach we hope to increase the overall satisfaction of the group by projecting the predicted satisfaction of those with the highest ratings for an item onto the rest of the group. We are disregarding accounting specifically for misery and dissatisfaction as it was discovered in \ref{d} \todo{Add citation to the article/study used in aggregation section} that such strategies performed poorly.

%Top 5 or Top 20\% (whichever is lower) of the ratings are multiplied.
%Rest is averaged and multiplied onto the result above.
\section{Conclusion and Future Work}\label{sec:conclusion}
Throughout this article we have strived to find the best performing method for aggregating individual ranked top-k lists into one top-k list suitable for a group of people.

%Overall BC performer best with MC as close second. 
What we found out was that BC in the most cases proves to be the superior approach and the cases where it is outperformed the different is very small and nearly insignificant. Worth noting is that MC's performance is almost as good as BC and further work could be done in trying to improve on these methods. 

%Interesting that BC performer better than MC and SF in group recommender when it is not the case in IR 
It was actually somewhat surprising that BC was the best performing because in a paper aggregating ranked search engine results it showed that BC was the worst, compared to SF and MF, and MF was the best.

%What measures to use and in what cases to use them?
Furthermore, we want to emphasize the importance of doing extensive testing as if we had limited our test we could easily have found that Avg or SF would have been the best methods. Specifically in the case with SF and SFD we can see how important to select an appropriate as well.\note{formulate this better}

That being said all the evaluation measures shows that as the groups gets bigger the decrease in the measures starts to fade out when the groups gets larger than 12 users. 
	%Measures follow the same pattern
	%Rating nDCG vs. nDCG	
\note{rating nDCG vs. nDCG }
\subsection{Future Work}

%survey/Dataset
\adparagraph{Measurements on real data}
To address the issue of nDCG as a satisfaction measure for group recommendations there is a need for real data. When training and evaluating our recommender system, we did not have any real data, and had to go through a number of assumptions for what makes a good group recommendation.

%Context and influence
\adparagraph{Context and Influence}
There has been work to document the effect of including influence and contextual information to improve recommendations for groups by people such as Quintarelli et al\cite{Quintarelli2016}. They show the scenario of a family unit sharing a television, and how the time of the day changes the influence of each family member.

%real world application


%Reordering of rank lists
\adparagraph{Reordering of Ranked Lists}
A pre-ranked aggregation method we did not test in our report was the reordering of the ranked lists.
The idea is to rearrange the rankings by the average rating from the other users before performing the aggregation step to account for the opinions of other users. It is our hypothesis that it would lead to better group satisfaction overall.

\section{Conclusion and Future Work}\label{sec:conclusion}
Throughout this article we have strived to find the best performing method for aggregating individual ranked top-k lists into one top-k list suitable for a group of people.

%Overall BC performer best with MC as close second. 
What we found out was that BC in the most cases proves to be the superior approach and the cases where it is outperformed the different is very small and nearly insignificant. Worth noting is that MC's performance is almost as good as BC and further work could be done in trying to improve on these methods. 

%Interesting that BC performer better than MC and SF in group recommender when it is not the case in IR 
It was actually somewhat surprising that BC was the best performing because in a paper aggregating ranked search engine results it showed that BC was the worst, compared to SF and MF, and MF was the best.

%What measures to use and in what cases to use them?
Furthermore, we want to emphasize the importance of doing extensive testing as if we had limited our test we could easily have found that Avg or SF would have been the best methods. Specifically in the case with SF and SFD we can see how important to select an appropriate as well.\note{formulate this better}

That being said all the evaluation measures shows that as the groups gets bigger the decrease in the measures starts to fade out when the groups gets larger than 12 users. 
	%Measures follow the same pattern
	%Rating nDCG vs. nDCG	
\note{rating nDCG vs. nDCG }
\subsection{Future Work}\label{sec:futurework}
%survey/Dataset 
%Context and influence 
%real world application
%Reordering of rank lists

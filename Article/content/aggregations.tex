\section{Group Recommendation}\label{sec:grouprecommendation}
This section documents the preprocessing done and outlines the rank aggregation methods used in order make the recommendation aggregation into a group recommendation. 
%In this section we explain the workings and concept behind the four aggregation methods.

\subsection{Preprocessing}
\note{Rewrite this to preprocessing}
Ranking is the idea of the position of an item in a ranked list is important. This means that if there is a list of ratings $\tau$ the list is ranked if $\tau (1) > \tau (2) > ... > \tau (k)$

When we talk about a top-k list in this paper we are referring to an ranked list of user preferences which is ordered by their item ratings in descending order.

There are several reasons for the use of ranked lists. 
The first reason is that we want to return a ranked list as a result of the aggregation since we want to ease the decision-making without making the final decision.
Secondly, there are a good selection of measurements for the quality of a ranked list. Many of these measures take two lists, $\tau_1$ and $\tau_2$, which are both ranked lists of the same size. 

In our case one of the lists is a top-k list and the second list is the ranked list of recommendations.
\subsection{Rank Aggregation}\label{sec:aggregations}
\{introduction?}
\begin{frame}[t]
\frametitle{Borda Count}
\begin{columns}
\begin{column}{0.5\textwidth}
\begin{align*}
	bc(i) = \sum_{u\in U} \tau_u(i)
\end{align*}
\small
\begin{align*}
	bc(A) = \tau_{u_1}(A) + \tau_{u_2}(A) + \tau_{u_3}(A) + \tau_{u_4}(A)
\end{align*}
\normalsize
\begin{align*}
	bc(A) = 5 + 4 + 2 + 3 = 14
\end{align*}

\end{column}
\begin{column}{0.5\textwidth}
\small
\vspace{-0.5cm}
\begin{table}
\captionsetup{font=footnotesize}
\begin{tabular}{|l|lllll|} \hline
Rank  & 1 & 2 & 3 & 4 & 5 \\\hline
$u_1$ & A & B & C & D & E \\
$u_2$ & C & D & F & A & E \\
$u_3$ & E & A & G & B & D \\
$u_4$ & G & H & A & E & F\\\hline
\end{tabular}
\caption{Top-k list of a group with 4 users}
\end{table}

\vspace{-1cm}
\begin{table}
\begin{tabular}{|l|llllllll|}\hline
      & A & B & C & D & E & F & G & H \\\hline
$u_1$ & 5 & 4 & 3 & 2 & 1 & 0 & 0 & 0 \\
$u_2$ & 2 & 0 & 5 & 4 & 1 & 3 & 0 & 0 \\
$u_3$ & 4 & 2 & 0 & 1 & 5 & 0 & 3 & 0 \\
$u_4$ & 3 & 0 & 0 & 0 & 2 & 1 & 5 & 4 \\\hline
Sum	  & 14& 6 & 8 & 7 & 9 & 4 & 8 & 4 \\\hline
\end{tabular}
\caption{Borda Count scores}
\end{table}
\normalsize
\end{column}
\end{columns}
\begin{table}
\begin{tabular}{llllll}
Recommendations: & A & E & C & G & D
\end{tabular}
\end{table}
\end{frame}

%
%\begin{columns}
%\begin{column}{0.5\textwidth}
%
%\end{column}
%\begin{column}{0.5\textwidth}
%
%\end{column}
%\end{columns}
\subsection{Markov Chain}\label{sec:markovchain}
The proposed Markov Chain method by Dwork et Al, \MC is a generalization of the Copeland Method\note{find cite for this}, where a winner is the candidate which wins the most pairwise contests\citep{rank:aggregation}.

The concept behind building the list of recommendations works by explicitly finding the transition matrix. For \MC, the states are connected to other states that wins per the Copeland method. Then we can iterate through the set, and note who performs best to make the transition matrix. Using the power set on the transition matrix we can find the stationary probability distribution to aggregate the candidates.

The \MC state space corresponds to a set of all the items ranked. The corresponding transition matrix for \MC will have an equal chance of transitioning to any other state that can beat it in a majority of pairwise contests.

Given partial lists $\tau_1,...,\tau_k$, collectively known as $\tau$, with rankings of items, and the state space, $S$, of \MC corresponding to the set of all items ranked in those partial lists. If the current state is item $i$ we can transition to uniformly picked state $j \in S$ where $j$ is ranked higher than item $i$ on a majority of lists in $\tau$ which ranked both $i$ and $j$. Otherwise, we stay in state $i$.

%Non-strict markov chain
\MC as presented by Dwork et al is used on metasearch and aggregating query results, whereas we work in the recommendation domain. To better suit our domain, we make a small adjustment to the method. For search engine comparison, partial lists might not contain both items needed for a pairwise comparison, so in the event of only one item being on the list, it is unknown if the other search engines have ranked the item or not, so Dwork et al restrict themselves to the pairwise comparisons available, and rely on the connectivity in the chain to correct any outcomes.

For our domain we have estimated the rankings giving us a top-k of a full list. So we follow this line of thinking for partial lists containing neither of the items, as the highest rank is not available. However, if the partial list contains one of the items, it wins that pairwise contest, as the losing item is known to be somewhere down the list.

%$p_{ij} = Pr(X_{n+1} = \sum_{l=1}^{k} \tau_l(i) < \tau_l(j) | X_n = \sum_{l=1}^{k} \tau_l(i) > \tau_l(j) )$

%Strict markov chain
%For completeness, we also tested a stricter interpretation of \MC where only the majority winners with both items present were considered.
\subsection{Spearman Footrule}\label{sec:spearmanfootrule}
\subsection{Average}\label{sec:average}
As a control algorithm we choose Average(Avg) aggregation as it is one of the more common used and well performing methods within group recommendation.\note{cite} The implementation we have done does only work with the items in the top-k list. It finds the union of all the users $u\in U$ lists so $\tau_1 \cup ... \cup \tau_u = I$. The Avg method then uses the full lists, $\sigma_1, ..., \sigma_u$, from the individual recommendations to find the average rating for the items $i \in I$. Equation \ref{eq:avg} illustrates how Avg works.

\begin{equation}\label{eq:avg}
Avg(i) = \frac{\sum_{u \in U} \sigma_u(i)}{|U|} 
\end{equation}
\note{ u in U doesn't stand under sum}
\subsection{Future Work}

%survey/Dataset
\adparagraph{Measurements on real data}
To address the issue of nDCG as a satisfaction measure for group recommendations there is a need for real data. When training and evaluating our recommender system, we did not have any real data, and had to go through a number of assumptions for what makes a good group recommendation.

%Context and influence
\adparagraph{Context and Influence}
There has been work to document the effect of including influence and contextual information to improve recommendations for groups by people such as Quintarelli et al\cite{Quintarelli2016}. They show the scenario of a family unit sharing a television, and how the time of the day changes the influence of each family member.

%real world application


%Reordering of rank lists
\adparagraph{Reordering of Ranked Lists}
A pre-ranked aggregation method we did not test in our report was the reordering of the ranked lists.
The idea is to rearrange the rankings by the average rating from the other users before performing the aggregation step to account for the opinions of other users. It is our hypothesis that it would lead to better group satisfaction overall.
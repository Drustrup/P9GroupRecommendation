\subsection{Future Work}\label{sec:futurework}

%survey/Dataset
\adparagraph{Measurements on real data}
To address the issue of nDCG as a satisfaction measure for group recommendations there is a need for real data. When training and evaluating our recommender system, we had to make do with individual ratings and make assumptions about what makes for good measures. With data on how people make recommendations, we could make some more informed conclusions on the used measures for group recommendations. There are several ways one could go in acquiring data. The first is to test group recommender methods on people, and have them give feedback on the results of the recommendation. This provides an indication of how the method performs for the group by aggregating the individual scores. Another way is to have people make recommendations based on information given about a group and test how close various methods come to these. The latter assumes that humans make good group recommendation systems and consistent about fairness, but it is also easier to generate large amounts of data on.\note{Check Mastoff's test setup for whether it could be talked about here}

%Test other BC and MC extensions
\adparagraph{BC and \MC extensions}
BC and \MC exist in many variants, and repeating our experiments with other extensions can reveal more about the measures and the extensions. Candidates could be other extensions presented by Dwork et al\cite{rank:aggregation} for Markov Chain, or others for Borda Count.

%Look into New Predicted Items vs Known Rating Items for movies
%\adparagraph{Prediction versus known items}
%For our work here, we used a recommender system to predict the ratings for items. 

%Context and influence
\adparagraph{Context and Influence}
Research on the effect of including influence and contextual information to improve recommendations for groups has been done by Quintarelli et al\cite{Quintarelli2016}. They show the scenario of a family unit sharing a television, and how the time of the day changes the influence of each family member.\note{Specify the theory used and explain how it relates to our project.}

%real world application
%Partially related to obtaining a dataset. Future work would be optimizing the algorithm for use

%Reordering of rank lists
\adparagraph{Reordering of Ranked Lists}
A pre-ranked aggregation method we did not test in our report was the reordering of the ranked lists.
The idea is to rearrange the rankings by the average rating from the other users before performing the aggregation step to better account for the opinions of other users. It is our hypothesis that it would lead to better group satisfaction overall in the case that Rating nDCG is a good measure.
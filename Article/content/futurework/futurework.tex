\subsection{Future Work}\label{sec:futurework}

%survey/Dataset
\adparagraph{Measurements on real data}
To address the issue of nDCG as a satisfaction measure for group recommendations there is a need for real data. When training and evaluating our recommender system, we had to make do with individual ratings and make assumptions about what makes for good measures.

With data on how people make recommendations, we could make some more informed conclusions on the used measures for group recommendations.

%Test other BC and MC extensions
\adparagraph{BC and \MC extensions}
BC and \MC exist in many variants, and repeating our experiments with other extensions can reveal more about the measures and the extensions.

Candidates Dwork et al had three other Markov Chain extensions. Given the difference in domain, there is room for 

%Look into New Predicted Items vs Known Rating Items for movies

%Context and influence
\adparagraph{Context and Influence}
There has been work to document the effect of including influence and contextual information to improve recommendations for groups by people such as Quintarelli et al\cite{Quintarelli2016}. They show the scenario of a family unit sharing a television, and how the time of the day changes the influence of each family member.

%real world application
%Partially related to obtaining a dataset. Future work would be optimizing the al

%Reordering of rank lists
\adparagraph{Reordering of Ranked Lists}
A pre-ranked aggregation method we did not test in our report was the reordering of the ranked lists.
The idea is to rearrange the rankings by the average rating from the other users before performing the aggregation step to account for the opinions of other users. It is our hypothesis that it would lead to better group satisfaction overall.
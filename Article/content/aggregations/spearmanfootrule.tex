\subsection{Spearman's Footrule}\label{sec:spearmansfootrule}
Dwork et al propose Spearman's footrule(SF) for aggregating ranked lists\citep{rank:aggregation}.
SF utilizes bipartite graphs from graph theory to construct a weighted complete bipartite graph $(S,P,W)$.
Let $S$ be the set of items equal to the union of the top-k lists $\tau_1, ..., \tau_u$. Then we have the set $P = \{1,...,k\}$, which are the available positions in the list to be recommended. Lastly, the set $W$ is the set of edge weights between items $s\in S$ and positions $p\in P$. The weights $W(s,p)$ are found by using the scaled footrule distance equation which can be seen in Equation \ref{eq:spearmanfootrule}\cite{rank:aggregation}.
 
\begin{equation}\label{eq:spearmanfootrule}
W(c,p) = \displaystyle\sum_{i=1}^{k} |r_i(c)/|r_i| - p/k|
\end{equation}

As we work with partial lists we will encounter lists with missing items. For this reason we have added a second case in addition to the approach described by Dwork et al, which can be seen in Equation \ref{eq:spearmanfootruleempty}. In this case we adapt the variable $\ell$ from Spearman's footrule distance. This variable is used on partial lists for measuring distance. $\ell$ needs to be larger than $k$ and in our case it is $k + 1$. The reason for this is to punish infrequent items by giving them a higher weight.

\begin{equation}\label{eq:spearmanfootruleempty}
W(c,p) = \displaystyle\sum_{i=1}^{k} |(\ell/|r_i| - p/k|
\end{equation}

After determining the edge weights, the problem can be solved as a minimum cost maximum matching problem, which is the problem of finding the highest number of node matches with the lowest edge cost. To do this, we decided to use the Munkres extension of the Hungarian method\cite{Munkres1957}. The result of this method is a ranked list of size $k$ containing the recommended items.
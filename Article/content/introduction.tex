\section{Introduction}
Many of the decisions we make are based on recommendations, from either people we know or recommender system tailored to personal preferences. This can be helpful due to the high amounts of information we process in our everyday lives\cite{Edwards2001}. The recommendations, or more specifically in our case, the recommender system, can cut down the number of options to a manageable level and thereby augment the decision-making process without forcing a decision.

The problem with the traditional recommender systems is that they typically make recommendations tailored to one person but often these decisions needs to be taken in a social context. 

For some scenarios, such as for selecting a movie on a streaming service, finding a restaurant, or deciding on a vacation destination, the inclusion of a social context would change the problem from that of knowing ones own preferences to that of the entire group in the given context.

One of the problems regarding taking the social context into consideration is that the recommender has to strive for consensus between the people it recommends to. An already complex problem is made even harder by having to solve it for multiple users simultaneously with new rules in play. From here, we will reference to this problem as making a group recommendation.

%Different methods for group recommendations 
When making group recommendations there are two main approaches, namely profile aggregation and recommendation aggregation\cite{profilvsrec}. The idea behind profile aggregation is to aggregate the users' preferences into a single group profile and make aggregations based on that profile. The other approach is to consider each user individually and aggregate the recommendations for the users into one aggregation that fits the groups preferences. In this paper we have chosen to focus on aggregation recommendation.

%as we want to make the method work for shifting groups and we deem this approach to fit this case best. \note{find source or fitting argument, Lukas: not sure there are any}

%Describe the approach of using top-k lists
As we are going to aggregate the users recommendations we have chosen to only focus on the top-k part of their recommendations and return a list of recommendations of size $k$ as a result. Furthermore, the top-k lists will be ranked with the highest rated item at first position on the list.

With ranked top-k lists being partial lists, we have selected three types of aggregation list which have shown good results when used for aggregating search engine results within the information retrieval domain for partial lists. The methods we are using is Borda Count, Markov Chain, and Spearman's Footrule\cite{Masthoff2004, rank:aggregation}. We also implemented an Average aggregation method as a control algorithm\cite{Masthoff2004}.

%Problem regarding evaluation of the aggregation results
For group recommendations we faced the challenge of evaluating the result without a dataset to provide a ground truth. However, from the information retrieval domain, we found measures to evaluate the quality of queries that can be used to evaluate the quality of a partial list of recommendations and there are many datasets available for individual recommendations.

%Baltrunas
For one such dataset, there have been made some discoveries for group recommendation. With some overlap with us in chosen methods, such as Borda Count, and measures, such as nDCG, Baltrunas et al found that the quality of the recommendation did not always drop even when the group size grows. However, their biggest group size was 8, and it was unclear if the trend could be seen for bigger group sizes\note{Check here, if you want to imply we're looking for this or whether the quality stops decreasing for increasing groupsizes}.

%We find out whether this is the case

\subsection{Research Questions}\note{extend and specify + more math}
Among common aggregation methods, given ranked top-k lists $\tau_1, ... , \tau_u$, where $u$ is the number of group members, which method can provide the most optimal group recommendation per measures such as satisfaction or distance from individual preferences of the group?

Also, is it the case for all aggregation methods that as the number of users, $u$, increases, does the quality of the recommendations decrease per our measures? For a set of groups, $G = \{ g_1, ..., g_n \}$ of corresponding group sizes, $U = \{ u_1, ..., u_n \}$, and group recommendation setup $f(x)$ that takes those groups as input and returns an evaluation of that recommendation, can we find a group size such that $f(g_a) > f(g_b)$ where $u_a > u_b$\note{Correct this part}.

%Find if derivative of f(g_1) = derivative of f(g_2), for any u_n where u_1 < u_2

\subsection{Structure of the Paper}\note{remember to adjust this}
The structure of the paper is as follows. Section \ref{sec:preliminaries} contains the preliminaries including the measures used during the evaluation. Section \ref{sec:aggregations} describes the aggregation methods used. In Section \ref{sec:evaluation} the performances of the aggregation methods are documented. Section \ref{sec:discussion} we discuss the results of the evaluation and in Section \ref{sec:conclusion} we will present our conclusion and future work.
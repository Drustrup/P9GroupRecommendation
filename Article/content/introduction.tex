\section{Introduction}
As it is no longer of question of should the computers be involved when humans make decisions but how.

Recommender systems can strengthen decision making without taking away the final choice from humans. This opens the 

For making a decision, Edwards et Al provide a 19 step guide for picking the option with the highest utility, and they argue that the problem for a user would be in picking from among the many options presented, also known as information overload\cite{Edwards2001}.

Recommender systems deal with reducing the problem for a user to a manageable amount of choices. Given a user's preferences, a good recommender system can narrow down the number of choices to a manageable level.

However when the problem area is expanded to include multiple users in need of a single choice, the problem area is two-fold, as the many individual preferences must be aligned into that of a single recommendation.

The recommender system is doubly challenged as while a single user can navigate the given recommendations and reflect on each item for the best optimal choice, a group will have a hard time making insightful decisions about items they lack the shared information of the group to comment on.
\note{Lukas: Possible direction for the introduction?}
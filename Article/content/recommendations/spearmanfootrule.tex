\subsection{Spearman Footrule}\label{sec:spearmanfootrule}
\note{early draft}
Spearman footrule rank aggregation creates a bipartite graph $(I,P,E)$. 
$I$ is the union of users top-k lists and $P$ is the total number of positions equal to $|I|$. $E$ is the weight between an item $i \in I$ and a position $p \in P$. In Equation \ref{eq:spearmanfootrule} it is shown how the weights is calculated for full lists. 

%where the weight on the edges in $E$ is the footrule distance from note u to position v. utilises the minimum cost maximum matching method to find the best possible ranked list based on a set $L$ of ranked lists. 

\begin{equation}\label{eq:spearmanfootrule}
W(c,p) = \displaystyle\sum_{i=1}^{k} |r_i(c)/|r_i| - p/n|
\end{equation}

In order to take items not appearing on some list, as we work with partial lists, into account, we make use of \ref{eq:spearmanfootruleempty} in order to weight infrequent items higher. 

\begin{equation}\label{eq:spearmanfootruleempty}
W(c,p) = \displaystyle\sum_{i=1}^{k} |(|r_i| + 1)/|r_i| - p/n|
\end{equation}

After determining the distances/weights the problem can be deduced to a minimum cost maximum matching problem in this case in the form of the Hungarian algorithm\note{section with the Hungarian algorithm?}. This method returns a list of length $|P|$ with to lowest edge weight as possible. We then return the top-k of the list as the groups recommendations.
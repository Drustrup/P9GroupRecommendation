\subsection{Dataset}\label{sec:dataset}
We use the dataset called MovieLens 100k published by GroupLens in 1998\cite{movielens100k}. MovieLens 100K contains 100.000 ratings between 1 to 5 collected from 943 users across 1682 movies.

With room for approximately one and a half million ratings, making the 100k rating dataset spars. So we will be using rating prediction in order to populate the rating matrix.

\subsubsection{Individual Recommendations}
For rating prediction, we used the library called MyMediaLite\cite{mymedialite}. MyMediaLite is a library for .NET that holds a bundle of recommendation methods for both item recommendation and rating prediction. We will be using the library, because this gives a tested foundation that's easy to reproduce for testing aggregation methods.

With SVD++ we can predict the missing ratings. For the parameters of SVD++, we used the settings shown in Table \ref{tbl:svdpp}.

\begin{table}[H]
	\centering
	\begin{tabular}{|l|l|}\hline
		Latent Factors & 50 \\
		Regularization & 1	\\
		Bias Regularization & 0.005	\\
		Learning Rate & 0.01 \\
		Bias Learning Rate & 0.07 \\ 
		Number of iterations & 50 \\
		Frequency Regularization & True \\ 
		Expected RMSE & 0.90651 \\
		Expected MAE & 0.71352 \\ \hline
	\end{tabular}
	\caption{Hyperparameters for the SVD++ component}
	\label{tbl:svdpp}
\end{table}

\subsubsection{Group Generation}\label{sec:groupgeneration}
Since voting systems perform differently depending on the number of users we made groups of several sizes.

For our dataset we made groups consisting of 4, 8, 12, 16, 20, and 40 users. Each size category numbers 1000 groups. The groups were randomly created, and while the same user can appear in multiple groups of various sizes, no two groups are identical.

Our expectations with this setup was that as the size of the group would go up, so would the difficulty of making recommendations. So it would be interesting to note whether all methods' recommendations decreased in quality at the same rate.
\subsubsection{Dataset}\label{sec:dataset}
We use the MovieLens 100k dataset published by GroupLens in 1998\cite{movielens100k}. MovieLens 100K contains 100.000 ratings between 1 to 5 collected from 943 users across 1682 movies.

With room for approximately one and a half million ratings, the 100k rating dataset is sparse. We will be using rating prediction in order to populate the rating matrix.

\subsubsection{Individual Recommendations}\label{sec:individualrecommendation}
For rating prediction, we used the library called MyMediaLite\cite{mymedialite}. MyMediaLite is a library for .NET that holds a bundle of recommendation methods for both item recommendation and rating prediction. We will be using the library, because this gives a tested foundation that is easy to reproduce for testing other aggregation methods.

Among the methods provided by MyMediaLite, SVD++ is one of the best performing on the 100k dataset on their own records\footnote{www.mymedialite.net/examples/datasets.html}. With SVD++ we can predict the missing ratings. For the parameters of SVD++, we used the settings shown in Table \ref{tbl:svdpp}.

\begin{table}[H]
	\centering
	\begin{tabular}{|l|l|}\hline
		Latent Factors & 50 \\
		Regularization & 1	\\
		Bias Regularization & 0.005	\\
		Learning Rate & 0.01 \\
		Bias Learning Rate & 0.07 \\ 
		Number of iterations & 50 \\
		Frequency Regularization & True \\ 
		RMSE & 0.90651 \\
		MAE & 0.71352 \\ \hline
	\end{tabular}
	\caption{Parameters for the SVD++ component}
	\label{tbl:svdpp}
\end{table}

\subsubsection{Group Generation}\label{sec:groupgeneration}
For our dataset we made groups consisting of 4, 8, 12, 16, 20, and 40 users. Given that the dataset contains 943 users, we will not have more than 5\% in any of the groups. As each size category numbers 1000 groups, 40 is more or less the highest we can go. The groups were randomly created, and while the same user can appear in multiple groups of various sizes.
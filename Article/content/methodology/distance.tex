\subsection{Distance}

\subsubsection{Kendall Tau Distance}
\note{This is a rough draft}
Kendall tau distance is a measure of the difference between the difference of ranked lists\citep{rank:aggregation}. It counts the pairwise disagreement between two ranked lists as specified in the Equations \ref{eq:kendalldistance1}, \ref{eq:kendalldistance2} and \ref{eq:kendalldistancefinal}. The count is the normalized by dividing with $n(n-1)/2$ where $n$ is the total number of available positions, giving the maximum number of possible values. \note{check up on this}
These two equations are used when working with full lists where every list contains every item. 

\begin{equation}\label{eq:kendalldistance1}
K1(\sigma,\tau) = | \{(i,j) | i < j, \sigma (i) < \sigma (j) \land \tau (i) > \tau (j)|
\end{equation}
\begin{equation}\label{eq:kendalldistance2}
K2(\sigma,\tau) = | \{(i,j) | i < j, \sigma (i) > \sigma (j) \land \tau (i) < \tau (j) \} |
\end{equation}

\begin{equation}\label{eq:kendalldistancefinal}
K(\sigma,\tau) = k2(\sigma,\tau) + k1(\sigma,\tau)
\end{equation}

As we are focusing on partial lists we have to make some modifications to the standard Kendall tau distance which we do by adding a third case in which it needs to count a disagreement. If a item occurs on one list but not the other it is assigned the value of the total number of items plus 1. We then add Equation \ref{eq:kendalldistance3} to Kendall tau distance in case we encounter equal items on one list which is undesirable.

\begin{equation}\label{eq:kendalldistance3}
K(\sigma,\tau) = | \{(i,j) | i < j, \sigma (i) = \sigma (j) \lor \tau (i) = \tau (j) \} |
\end{equation}

To find the distance from a recommended list to a groups preferences each list of user preferences are compared to the result list and the average distance is the result. 



%\subsubsection{Spearman Distance}
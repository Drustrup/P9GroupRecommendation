\begin{titlepage}
\section*{Summary}
Most commonly in recommendation, it is for a single person. The classic problem for the Recommendation System is to provide the best item from a variety of options to a single user. It has branched off into a multitude of branches such as collaborative and content-based filtering.

Today, recommender systems concern themselves about where we should eat, what music to listen to, what movie to watch, or where our next vacation should go to.

However, none of the scenarios above are uniquely activities done alone. Some are traditionally outright viewed as group activities for most people by default. As such, the concept of a group recommender is an intuitive extension to the traditional recommender.

This article deals with recommendation for groups of people. The problem is reflected in many other aspects of life and it is radically different from the challenges of a normal recommender system. Voting shares similarities with the challenges seen in group recommendation, as the challenge is to recommend the option that is the least opposed by all parties or satisfies some other criteria for approval in the group.

So instead of a recommendation, the challenge is counting votes. For Group Recommendation, this is usually defined as aggregation, and many aggregation methods exist and are used for many domains. Borda Count, which exists both as a voting and aggregation method, is one such example and is used in this master thesis project.

Borda Count in particular was notable for the project. In the previous semester, the group working on this project was exploring extensions of Borda Count. The results were promising, but there was a lack of a ground truth to really give meaning to the results.

For the project, initially, we were chasing the possibility of making a dataset establishing a ground truth. The dataset itself would have been a great contribution. However, given the amount of data needed for a proper dataset, we had to look towards paid services to attract the numbers needed. In this case, we turned to Amazon Mechanical Turk, where it is possible to pay people to answer surveys or other simple tasks.

The aim of the
\end{titlepage}
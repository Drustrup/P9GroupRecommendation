\section{Background}

\begin{frame}
        \centering
        \huge Background
        \note{
                \begin{itemize}
                	\item Twitter
                	\item Data Stream Model
                	\item Firehose
                \end{itemize}
                }
\end{frame}

\begin{frame}
	\frametitle{Contributions}
	\begin{itemize}
		\item Value of Twitter Streaming data
		\item Covering challenges of Twitter streaming data
		\item Sliding window Kappa Statistic
		\item Recommendation of a classifier
	\end{itemize}
	\note{
		\begin{itemize}
			\item There is a lot of value to be gained through twitter data, which has been recently been given
			\item Challenges: unbalanced set of data
			\item A statistic besides prequential accuracy
		\end{itemize}
		}
\end{frame}

\begin{frame}
	\frametitle{Twitter}
	\begin{itemize}
		\item 106 million users, 2010
		\item Firehose
		\item Data Stream Model
	\end{itemize}
	\begin{figure}
		\centering
		\includegraphics[scale=0.5]{datastream.png}
	\end{figure}
	\note{At a twitter conference, firehose was announced. It follows a data stream model - }
\end{frame}

\begin{frame}
	\frametitle{Twitter Streaming API}
	\begin{block}{JSON}
		"user":\{\\
		\quad "statuses\_count":3080,\\
		\quad "favourites\_count":22,\\
		\quad "name":"Twitter API",\\
		\quad "following":true,\\
		\quad "description":"The Real Twitter API. I tweet about API \\ \quad changes, service issues and happily answer questions\\ \quad  about Twitter and our API. Don't get an answer? It's on my \\ \quad website.",\\
		\quad "location":"San Francisco, CA"\\
		\}
	\end{block}
	\note{
		\begin{itemize}
			\item The twitter API returns results in JSON that looks like this
			\item This has been shortened
		\end{itemize}}
\end{frame}
\section{Influence and Context}
Quintarelli et al documented the positive effect of influence and contextual influence when recommending items for a group\cite{contextinfluence}. An example the article uses to explain contextual influence with, is the scenario of a family, consisting of parents and kids, selecting what to watch on the television. In this scenario they have two contexts, early afternoon and late afternoon. They then argue that in the early afternoon the children have a greater influence because most of the tv-programs are children friendly at this time. In the late afternoon the parents then get more influence because child inappropriate programs may occur. 

The idea of context influence could be applicable in many scenarios and the selection can be extensive. It could be interesting to apply this approach to our method in order to see if it would improve the user satisfaction, but in order to do this we would need a dataset over group preferences in order to detect any improvements. Furthermore, the dataset should contain some sort of context. 